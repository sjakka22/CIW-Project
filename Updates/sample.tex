\title{Unsere Erfahrungsberichte zum BioInf-Studium}
\subtitle{\& Zeit für Eure Fragen}

% \institute[Uni Hamburg]{Universität Hamburg · Fachbereich Informatik}
\date{} % Ausblenden geht mit leerem Inhalt: \date{}
% \titlegraphic{\uhhlogo} % Logo für Titelseite, UHH-Logo ist Standard
\eventlocation{Hamburg, \today} % rechts oben auf Titelseite
\setbeamertemplate{footline}[frame number]
\begin{document}

\begin{frame}
	% Die Titelseite erscheint nach erneutem Übersetzen korrekt.
	\maketitle
    
\end{frame}





\begin{frame}
    \frametitle{Allgemeines}
    \logos
    \begin{itemize}
        \item Materialsammlung: 
        \begin{itemize}
            \item https://mafiasi.de/gprot/
            \item https://cloud.mafiasi.de/apps/files/files/18800065?dir=/Bio\%20Inf\%20Cloud
        \end{itemize}
        \item Aufgezeichnete Vorlesungen, SSH-Einrichtung:
        \begin{itemize}
            \item https://lecture2go.uni-hamburg.de/
            \item https://www.zbh.uni-hamburg.de/studium/vortraege/aufzeichnungen.html
            \item Rubrik Service: SSH-Einrichtung
            \item ZBH Wiki: https://wiki.zbh.uni-hamburg.de/
        \end{itemize}
                \item Nutzt die Möglichkeit zur Gruppenarbeit!
        \item Klausuren: 2 Termine pro Semester - im WS: Februar \& März, kein Termin im Sommer
        \item Klausurtermine + An-/Abmeldungsfristen: \\
        \begin{itemize}
            \item https://www.inf.uni-hamburg.de/studies/orga/dates/2025-wise-written-exams.html
        \end{itemize}
        \item Studentische Beteiligung: Evaluation, Round-Table, Qualitätszirkel
    \end{itemize}
\end{frame}

\begin{frame}
     \frametitle{Studienverlaufsplan}
  \logos 
  \begin{figure}
      \centering
      \includegraphics[width=\textwidth]{tetris-msc-bioinf-2021.png}
  \end{figure}
\end{frame}

\begin{frame}
     \frametitle{PFN1 - Programmierung für Naturwissenschaften I}
  \logos 
    \begin{alertblock}{Inhalte}
  \small
      \begin{itemize}
          \item Umgang mit Linux
          \item \textbf{Systematische} Einführung in Python mittels naturwissenschaftlicher Fragestellungen
          \item Arbeit mit \texttt{numpy} und \texttt{matplotlib}
      \end{itemize}
 
  \end{alertblock}

\pause
Kommentare:
\begin{itemize}
    \item Zunächst vieles bekannt und einfach, dann viele neue Inhalte
    \item Folien selbsterklärend
    \item Übungen aufwendig, aber hier lernt man "wie man programmiert"
    \item \texttt{Make} files helfen bei der Bearbeitung
    \item Ausgabe eines Fragenkatalogs am Ende der VL
    \item Ausgabe von zusätzlichen Selbsttests (hilfreich!)
\end{itemize}
\end{frame}
\begin{frame}
     \frametitle{AD - Algorithmen und Datenstrukturen}
  \logos 

\begin{alertblock}{Inhalte}
    \begin{itemize}
        \item Einführung in Algorithmen, Laufzeitanalyse
        \item Sortieren, Suchen, Graphalgorithmen
        \item Weitere Entwurfsmethoden, Schwere Probleme
    \end{itemize}
\end{alertblock}
\pause
Kommentare:
\begin{itemize}
    \item Begleitende Tutorien zur Übung? Hilfsmittel Klausur?
    \item 7 Übungstermine, zwei-wöchiger Turnus?
    \item Nachlesen im Cormen hilfreich
    \item Übungsblätter aufwendig - hilft aber beim Entwurf von Algorithmen
    \item Beweisführung und mathematische Notation wichtig
    \item Am wichtigsten: Unvoreingenommen in die Klausur gehen!
\end{itemize}

\end{frame}


\begin{frame}
     \frametitle{GSA - Grundlagen der Sequenzanalyse}
  \logos 
    \begin{alertblock}{Inhalte}
  \small
      \begin{itemize}
          \item Modell der Edit-Distanz für die biologische Sequenzanalyse, Alignments
          \item Aligmentfreie Sequenzvergleiche
          \item Multiple Sequenzaligments
      \end{itemize}
 
  \end{alertblock}

\pause
Kommentare:
\begin{itemize}
    \item viele Definitionen, Formeln, Beweise
    \item betont mathematisch-informatisch, statt biologisch
    \item Oft Brücken zu \textbf{AD} und \textbf{PFN1}
    \item Übungen ab Woche 4 / 5 als Python-Implementierungen
    \item Folien selbsterklärend
    \item \texttt{Make} files helfen bei der Bearbeitung
    \item Ausgabe des Fragenkatalogs am Ende der VL
\end{itemize}


  
\end{frame}


\begin{frame}
     \frametitle{GSB - Grundlagen der Systembiologie}
  \logos 
    \begin{alertblock}{Inhalte}
  \small
      \begin{itemize}
          \item Molecular biology, algorithms
          \item Biostatistics, Genomics, Epigenetics, Transcriptomics
          \item Sequence motif prediction \& de novo discovery, Markov models, HMM, Machine learning
      \end{itemize}
  \end{alertblock} 
\pause 
Kommentare:
\begin{itemize}
    \item Viele Themen, ein Thema pro VL
    \item Übungen gut zu bewältigen (3-5 Personen pro Gruppe)
    \item weniger mathematisch-informatisch, stärker biologisch, anwendungsbezogen, Beispiele aus der Forschung
    \item Im Vorjahr: Präsentation in Kleingruppen zu ausgewähltem Thema / Publikation
\end{itemize}
\end{frame}


\begin{frame}
     \frametitle{GCI - Grundlagen der Chemieinformatik}
  \logos 

    \begin{alertblock}{Inhalte}
  \small
      \begin{itemize}
          \item Datenbanken, Moleküldarstellung und -repräsentation, Mustersprachen
          \item Algorithmen d. Chemieinformatik (Substruktursuche)
          \item Molekulare Ähnlichkeit
      \end{itemize}
  \end{alertblock}
\pause
Kommentare:
\begin{itemize}
    \item Auch Datenbanken sind Teil der Klausur
    \item Viele Bezüge zu \textbf{AD} (Graphalgorithmen)
    \item Wöchentliche Übungen, Gruppe aus max. 3 Personen
    \item GCI ist betont algorithmisch, CIW (im Sommer) mit Fokus Computer-Aided Drug Design
    \item C++ Übergangskurs mit kurzem Test Anfang März?
\end{itemize}


\end{frame}

\begin{frame}
     \frametitle{GST - Grundlagen der Strukturanalyse}
  \logos 


    \begin{alertblock}{Inhalte}
  \small
      \begin{itemize}
          \item Untersuchung von Protein-Strukturen - vom Experiment zu 3D-Koordinaten
          \item Modellierung von Protein-Strukturen
          \item Ähnlichkeit von 3D-Strukturen
      \end{itemize}
  \end{alertblock}
\pause
Kommentare:
\begin{itemize}
    \item Vorlesungsfolien knapper gehalten, eigene Notizen wichtig
    \item Übung: viele kleine Anwendungsaufgaben
    \item Revision- und Testfragen im Moodle-Kurs
    \item Workload gut zu bewältigen
\end{itemize}


  
\end{frame}






\end{document}