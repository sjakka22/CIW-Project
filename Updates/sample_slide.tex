\begin{frame}{Projektstruktur}
(vielleicht als tikz Image)
\begin{itemize}
    \item Input: 
    \begin{itemize}
        \item co-kristallisierte Protein-Ligand-Komplexe (PDB), insbesondere muss der Bindungsmodus \textbf{nicht} gefunden werden?!
        \item 3D-Strukturen kleiner organischer Moleküle (CSD)
    \end{itemize}
    \item Output: 
    \begin{itemize}
        \item Visualisierung von PL-Interaktionen (wie bei PoseView, Maestro, PLIP) \\
        $\Rightarrow$ 2D oder 3D?
        \item Parametrisierung, d. h. der Nutzer soll das Interaktionsmodell definieren.\\
        $\Rightarrow$ Wollen wir basierend auf CSD ein geeignetes Interaktionsmodell vorschlagen, was die maximale Zahl an Interaktionen vorschlägt? Enthalpische Optimierung
        \item Ableitung von Interaktionsgeometrien aus PDB und CSD
        $\Rightarrow$ Beispiel: Wie ist die Verteilung des DHA-Abstand einer hydrogen bond für N-H-O? Können wir anhand dieser Verteilung in TP1 gefundene Interaktionen bewerten / scoren?
    \end{itemize}
\end{itemize}

\end{frame}